\documentclass{article}

%%%%%%%%%%%%%%%%%%%%%%%%%%%%%%%%%%%%%%%%%%%%%%%%%%%%%%%%%%%%%%%%%%%%%%%

\usepackage{fullpage}
\usepackage{url}
\usepackage{graphicx}
\usepackage{color}
\definecolor{javared}{rgb}{0.6,0,0}
\definecolor{javagreen}{rgb}{0.25,0.5,0.35}
\definecolor{javapurple}{rgb}{0.5,0,0.35}
\definecolor{javadocblue}{rgb}{0.25,0.35,0.75}
\usepackage{listings}
\lstset {
	language=Java,
	basicstyle=\ttfamily,
	keywordstyle=\color{javapurple}\bfseries,
	stringstyle=\color{javared},
	commentstyle=\color{javagreen},
	morecomment=[s][\color{javadocblue}]{/**}{*/},
	numbers=left,
	numberstyle=\color{black},
	tabsize=4,
	showspaces=false,
	showstringspaces=false
}
%\usepackage{fullpage}

% Remove red border around refs and make them stand out.
\usepackage[colorlinks=true,linkcolor=blue]{hyperref}

%%%%%%%%%%%%%%%%%%%%%%%%%%%%%%%%%%%%%%%%%%%%%%%%%%%%%%%%%%%%%%%%%%%%%%%
%% Check these macro values for appropriateness for your own document.

\title{IS3 Group Report}

%%authors
\author{
  Richard Fleming \\
  James Gallagher \\
  Craig McLaughlin \\
  Victor Pantazi \\
  Gordon Reid \\
  Ross Taylor}

%%release date 
\date{\today}

%%%%%%%%%%%%%%%%%%%%%%%%%%%%%%%%%%%%%%%%%%%%%%%%%%%%%%%%%%%%%%%%%%%%%%%

\begin{document}

%%%%%%%%%%%%%%%%%%%%%%%%%%%%%%%%%%%%%%%%%%%%%%%%%%%%%%%%%%%%%%%%%%%%%%%

\maketitle

%%%%%%%%%%%%%%%%%%%%%%%%%%%%%%%%%%%%%%%%%%%%%%%%%%%%%%%%%%%%%%%%%%%%%%%
%% Standard section for all documents

\section{Overview}

Basic stuff about what we did...

\section{Specification}
The swing prototype does not meet every detail in the specification.
        The following list displays the current functionality with shortcomings,
        when compared to the specification, that is currently included in the
        prototype:
        \begin{itemize}
       
        \item It is possible for the user to add and delete appointments. This is
        done by manipulating a text file; writes to the text file when creating an
         event and reading from the file when needed to retrieve events for deletion.
        \item Edit an appointment is not currently possible, although the user can
        always delete an event and re-create it to get around this problem.
        \item The route the user would take to set a recurring appointment is
        visually displayed however the functionality of mapping that appointment
        is not included.
        \item It's possible to switch between day, week ,month and year view
        via buttons that are perpetually located at the top of the calendar. Current
        system status is shown by which button is currently depressed.
        \item In order for the user to determine the busiest/quietest days, in year
        view each day has a cell and is coloured depending on busyness. There is a
        gradual change from red, on the busy side of scale, to blue, on the non busy
        side of the scale.
        \item  During the event creation process, it is possible for the user to
        select the category associated with the event, whether it be home, work or
        other. However, as with the recurring appointment, the mapping of category
        and events is not implemented.
        \item As the events are not mapped to a category it is not currently
        possible to display events of a given category.

        \end{itemize}
\section{Evaluation of design}

Including design of experience and what we found out...

\section{Comparison with paper prototype}

When implementing the Swing Prototype we first attempted to implement
the more essential features of our paper prototype then moving on to
implementing less important features.

We began by implementing the Month view we had in our paper prototype.
We decided to keep the majority of features the same as they appeared
in the paper prototype, the only exception being the summary side bar.
Initially we omitted the side bar, intending to implement it once more
important features had been implemented. However, in the interest of
simplifying the implementation process, we ultimately decided to
implement this feature as a separate 'Day' view. The Week and Year
views were then created the same as they appeared in the paper
prototype, with the addition of an another button at the top of the
screen to accomodate the new Day view.

Then we created the Dialogs for the pop up windows which appeared in our
paper prototype when the user was adding new events, adding event
reminders and adding repetitions to events. These were implemented with
the same features as they had in the paper prototype. We also
implemented the pop up window for setting event categories the same as
it appeared in the paper prototype but without the button for adding new
categories.

\appendix

\section{Code listing}

\subsection{BaseView.java}

\lstinputlisting{../BaseView.java}

\subsection{CalendarModel.java}

\lstinputlisting{../CalendarModel.java}

\subsection{CalendarOperation.java}

\lstinputlisting{../CalendarOperation.java}

\subsection{CategoryDialog.java}

\lstinputlisting{../CategoryDialog.java}

\subsection{Date.java}

\lstinputlisting{../Date.java}

\subsection{DateDialog.java}

\lstinputlisting{../DateDialog.java}

\subsection{DayDataModel.java}

\lstinputlisting{../DayDataModel.java}

\subsection{DayView.java}

\lstinputlisting{../DayView.java}

\subsection{Event.java}

\lstinputlisting{../Event.java}

\subsection{EventDialog.java}

\lstinputlisting{../EventDialog.java}

\subsection{EventRenderer.java}

\lstinputlisting{../EventRenderer.java}

\subsection{JEventField.java}

\lstinputlisting{../JEventField.java}

\subsection{Main.java}

\lstinputlisting{../Main.java}

\subsection{MainFrame.java}

\lstinputlisting{../MainFrame.java}

\subsection{MonthDataModel.java}

\lstinputlisting{../MonthDataModel.java}

\subsection{MonthView.java}

\lstinputlisting{../MonthView.java}

\subsection{Reminder.java}

\lstinputlisting{../Reminder.java}

\subsection{ReminderDialog.java}

\lstinputlisting{../ReminderDialog.java}

\subsection{Repetition.java}

\lstinputlisting{../Repetition.java}

\subsection{RepetitionDialog.java}

\lstinputlisting{../RepetitionDialog.java}

\subsection{SpacedPanel.java}

\lstinputlisting{../SpacedPanel.java}

\subsection{Time.java}

\lstinputlisting{../Time.java}

\subsection{WeekDataModel.java}

\lstinputlisting{../WeekDataModel.java}

\subsection{WeekView.java}

\lstinputlisting{../WeekView.java}

\subsection{YearDataModel.java}

\lstinputlisting{../YearDataModel.java}

\subsection{YearView.java}

\lstinputlisting{../YearView.java}

\end{document}

%%%%%%%%%%%%%%%%%%%%%%%%%%%%%%%%%%%%%%%%%%%%%%%%%%%%%%%%%%%%%%%%%%%%%%%
