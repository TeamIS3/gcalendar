\documentclass{article}

%%%%%%%%%%%%%%%%%%%%%%%%%%%%%%%%%%%%%%%%%%%%%%%%%%%%%%%%%%%%%%%%%%%%%%%

\usepackage{fullpage}
\usepackage{url}
\usepackage{graphicx}
\usepackage{color}
\definecolor{javared}{rgb}{0.6,0,0}
\definecolor{javagreen}{rgb}{0.25,0.5,0.35}
\definecolor{javapurple}{rgb}{0.5,0,0.35}
\definecolor{javadocblue}{rgb}{0.25,0.35,0.75}
\usepackage{listings}
\lstset {
	language=Java,
	basicstyle=\ttfamily,
	keywordstyle=\color{javapurple}\bfseries,
	stringstyle=\color{javared},
	commentstyle=\color{javagreen},
	morecomment=[s][\color{javadocblue}]{/**}{*/},
	numbers=left,
	numberstyle=\color{black},
	tabsize=4,
	showspaces=false,
	showstringspaces=false
}
%\usepackage{fullpage}

% Remove red border around refs and make them stand out.
\usepackage[colorlinks=true,linkcolor=blue]{hyperref}

%%%%%%%%%%%%%%%%%%%%%%%%%%%%%%%%%%%%%%%%%%%%%%%%%%%%%%%%%%%%%%%%%%%%%%%
%% Check these macro values for appropriateness for your own document.

\title{IS3 Group Report}

%%authors
\author{
  Richard Fleming \\
  James Gallagher \\
  Craig McLaughlin \\
  Victor Pantazi \\  
  Gordon Reid \\
  Ross Taylor}

%%release date 
\date{\today}

%%%%%%%%%%%%%%%%%%%%%%%%%%%%%%%%%%%%%%%%%%%%%%%%%%%%%%%%%%%%%%%%%%%%%%%

\begin{document}

%%%%%%%%%%%%%%%%%%%%%%%%%%%%%%%%%%%%%%%%%%%%%%%%%%%%%%%%%%%%%%%%%%%%%%%

\maketitle

%%%%%%%%%%%%%%%%%%%%%%%%%%%%%%%%%%%%%%%%%%%%%%%%%%%%%%%%%%%%%%%%%%%%%%%
%% Standard section for all documents

\section{Overview}

This report describes the design, development and evaluation of our 
Java Swing Calendar application. To implement the Swing prototype, the
Model View Controller architecture was used. In the system the Model
is represented by a mapping of the events currently saved in the
system to the dates on which they occur.\\The system has multiple views.
One for `Week', one for `Month' and one for `Year'. The Month view 
aquires information about the events it needs to display from the
Model. The Controllers are represented by the Action Listeners
as they are the parts of the system which listen for user input (button
clicks).

\subsection{Views}

The first thing to be implemented was a Base View class which is an
extension of a JPanel. This class contains things which all views need
such as various Swing components, information about the dates which
the view must display and variables to store the Model data. All view
classes, therefore, extend this class and add specific things required
only by that specific view.

The `Month' view contains a JTable to represent the month with each cell
in the table representing a day in the month. This view also contains a
JPanel at the side of the page which was intended to act as the summary
side bar from the paper prototype. However, this was not implemented in
time so it simply displays the hours in a certain day but shows no
events.

The `Week' view also contains a table. Here, the table has 8 columns and
24 rows. 7 of the columns represent the days of the week and the
remaining column shows an hour of the day. Therefore, each cell
represents one hour in a certain day of the week.

The `Year' view contains a 4x3 grid which holds 12 tables to represent
months of the year. The cells of the tables each display a number to
show which day of the month the cell represents.

\subsection{Pop-up Windows}

The pop up windows which were used in the paper prototype when users
were attempting to complete certain tasks were also implemented in the
Swing prototype and, for the most part, remained unchanged from those
seen in the paper prototype. The implementation of these was largely
straightforward with the windows mainly consisting of simple Swing
components such as JButtons, JRadioButtons, JTextFields etc.

The windows for users setting repetitions, setting reminders and
adding/editing/deleting events are all implemented using two classes: a
Dialog class and a class to represent the variable being created by the
user in the window. The Dialog class controls the display of the window
and Implements ActionListener to allow the user to enter data into the
window.

%%%%%%%%%%%%%%%%%%%%%%%%%%%%%%%%%%%%%%%%%%%%%%%%%%%%%%%%%%%%%%%%%%%%%%%%

\section{Specification}

The swing prototype does not meet every detail in the specification.
The following list displays the current functionality with shortcomings,
when compared to the specification, that is currently included in the
prototype:

\begin{itemize}
       
\item It is possible for the user to add and delete appointments. This is
done by manipulating a text file; writes to the text file when creating an
event and reading from the file when needed to retrieve events for deletion.
\item Editing an appointment is currently possible, although the user can
only edit the last event that was created.
\item The route the user would take to set a recurring appointment is
visually displayed however the functionality of mapping that appointment
is not included.
\item It is possible to switch between week, month and year view
via buttons that are perpetually located at the top of the calendar. Current
system status is shown by which button is currently depressed.
\item  During the event creation process, it is possible for the user to
select the category associated with the event, whether it be home, work or
other. However, as with the recurring appointment, the mapping of category
and events is not implemented.
\item As the events are not mapped to a category it is not currently
possible to display events of a given category.

\end{itemize}

%%%%%%%%%%%%%%%%%%%%%%%%%%%%%%%%%%%%%%%%%%%%%%%%%%%%%%%%%%%%%%%%%%%%%%%%

\section{Evaluation of design}

To evaluate the design of the Swing prototype that the group produced, two main techneques were used.  

The first method the group decided to use for evaluating the design of the prototype was an heuristic evaluation. 
The group felt that a heuristic evaluation is a fast and cheap way to preform a usability inspection on the prototype, and would be an easy way to identify usability problems with the design of the system. 

One member of the team (James) was selected as an evaluator, and tasked with examing the interface of the prototype calendar and evaluating how it complies with recognized usability principles. 

The first thing the evaluator did was to select a collection of use cases, which he could then use to evaluate the extent to which the interface is likely to be compatible with the intended users’ needs. 

The use cases selected were : 
\begin{itemize}
\item 1.Change views from month, to week, to day 
\item 2.Add an appointment 
\item 3.Set a recurring appointment to occur once a week and finish in three months time
\item 4.Find the busiest and quietest days in the month
\item 5.Set different categories for university, social and job appointments
\end{itemize}

Afer this the evaluator decided on a set of heuristics on which the prototype should be judged. 

\subsection{Heuristic Evaluation}

The evaluator decided upon a set of heuristics on which the Swing prototype the team built should be judged.

The heuristics selected were as follows : 
\begin{itemize}
\item Visibility of system status : The system should always keep users informed about what is going on, through appropriate feedback within reasonable time. 

\item Match between system and the real world : The system should speak the users' language, with words, phrases and concepts familiar to the user, rather than system-oriented terms, making information appear in a natural and logical order.
 
\item User control and freedom : Users often choose system functions by mistake and will need a clearly marked "emergency exit" to leave the unwanted state without having to go through an extended dialogue. Support undo and redo. 

\item Consistency and standards : Users should not have to wonder whether different words, situations, or actions mean the same thing. Follow platform conventions. 

\item Error prevention : Even better than good error messages is a careful design which prevents a problem from occurring in the first place. Either eliminate error-prone conditions or check for them and present users with a confirmation option before they commit to the action. 

\item Recognition rather than recall : Minimize the user's memory load by making objects, actions, and options visible. The user should not have to remember information from one part of the dialogue to another. Instructions for use of the system should be visible or easily retrievable whenever appropriate. 

\item Flexibility and efficiency of use : Accelerators -- unseen by the novice user -- may often speed up the interaction for the expert user such that the system can cater to both inexperienced and experienced users. Allow users to tailor frequent actions. 
    
\item Aesthetic and minimalist design :Dialogues should not contain information which is irrelevant or rarely needed. Every extra unit of information in a dialogue competes with the relevant units of information and diminishes their relative visibility. 
    
\item Help users recognize, diagnose, and recover from errors : Error messages should be expressed in plain language (no codes), precisely indicate the problem, and constructively suggest a solution. 
    
\item Help and documentation : Even though it is better if the system can be used without documentation, it may be necessary to provide help and documentation. Any such information should be easy to search, focused on the user's task, list concrete steps to be carried out, and not be too large.    
\end{itemize}

Below is a summary of the findings from the evaluation:

\begin{itemize}
\item \textbf{Add an appointment}
The option to add an event to the prototype calendar is clearly visible from all the views in the prototype from the menu bar.
Once selected, a clearly laid out window opens up with a simple and easy to understand UI. In the Event Dialog window the user is presented with two panels to enter the start and end date of their event. The user enters this information through pop-up windows with specific day/month and year drop down boxes to enter these details. During the evaluation, it was found that users took a long time to complete all the forms of the Event Dialog, most of which were not mandatory. It was suggested that a faster, simpler input method might be needed. 
\item \textbf{Change views from month, to week, to year}
The option to change views from month, to week, to year in the Swing prototype is clearly visible from all the views in the prototype. The option to change view is shown in the top right corner by 3 simple buttons. This task was
completed quickly and without issue by all participants.
\item \textbf{Set a recurring appointment to occur once a week and finish in three months time}
The option to add an event to the prototype calendar is clearly visible from all the views in the prototype from the menu bar.
Once selected the Event Dialog window opens up with a simple and easy to understand UI. This was easily found by all participants. In the appointment window the user is displayed with a set repetition option. A third window is then opened in front of the user giving the option to set the appointment to be repeated every Day/Week/Fortnight/Month/Year or per x days.
However, the repetitions were not actually implemented so, while all
participants followed the correct procedure and had no major problems,
no repeating event was actually displayed.
\item \textbf{Find the busiest and quietest days in the month}
Finding the busiest day in the month can only be done by entering month
view and counting the number of events which occur in each day. During
the evaluation this was found to be rather awkward and a number of
participants remarked that a something like a colour coding system or a
simple display of the number of events in a day would be more
user-friendly and convenient.
\item \textbf{Set different categories for university, social and job appointments}
The `Category' option is clearly visible on the menu bar from all views
and all participants found this option with ease and in a reasonable
time. When clicked, this option caused the pop-up Category Dialog window
to open. All participants then clicked the `Add New' button. However,
this feature was not implemented in time. This meant that when
participants attempted to click this option nothing happened and a
category was not created.
\end{itemize}

\subsection{Timing}

\begin{table}[ht]
\caption{Heuristic Evaluation timing results (seconds)} % title of Table
\centering % used for centering table
\begin{tabular}{c c c c} % centered columns (4 columns)
\hline\hline %inserts double horizontal lines
Case & User\#1 & User\#2 & User\#3 \\ [0.5ex] % inserts table
%heading
\hline % inserts single horizontal line
1 & 9.96 & 12.34 & 12.23 \\ % inserting body of the table
2 & 1.62 & 2.01 & 1.71 \\
3 & 17.54 & 12.20 & 15.67 \\
4 & 23.32 & N/A & 5.44 \\
5 & 22.85 & 18.93 & 26.42 \\ [1ex] % [1ex] adds vertical space
\hline %inserts single line
\end{tabular}
\label{table:nonlin} % is used to refer this table in the text
\end{table}

%%%%%%%%%%%%%%%%%%%%%%%%%%%%%%%%%%%%%%%%%%%%%%%%%%%%%%%%%%%%%%%%%%%%%%%%

\section{Comparison with paper prototype}

When implementing the Swing Prototype, the more essential features of the
paper prototype were implemented first, then work was started on the less
essential features.

The Month view from the paper prototype was implemented first. The
majority of features we implemented are the same as they appeared in the
paper prototype, the only exception being the summary side bar.
Initially the side bar was omitted from the Swing prototype to be
implemented once more important features had been completed.
The Week and Year views were then created the same as they appeared in 
the paper prototype.

Then the Dialogs for the pop up windows (which appeared in the paper
prototype when the user was adding new events, adding event
reminders and adding repetitions to events) were created. These were
implemented with the same features as they had in the paper prototype. 
The pop up window for setting event categories was then implemented the
same as it appeared in the paper prototype but without the button for 
adding new categories.

Due to time contsraints the decision was made not to implement the
`click-and-drag to edit event dates' feature. Due to the fact that the
ability to edit events on each view already existed within the standard
menu bar, implementing this feature for each different view was
unnecessary.

%%%%%%%%%%%%%%%%%%%%%%%%%%%%%%%%%%%%%%%%%%%%%%%%%%%%%%%%%%%%%%%%%%%%%%%%

\appendix

\section{Code listing}

\subsection{BaseView.java}

\lstinputlisting{../BaseView.java}

\subsection{CalendarModel.java}

\lstinputlisting{../CalendarModel.java}

\subsection{CalendarOperation.java}

\lstinputlisting{../CalendarOperation.java}

\subsection{CategoryDialog.java}

\lstinputlisting{../CategoryDialog.java}

\subsection{Date.java}

\lstinputlisting{../Date.java}

\subsection{DateDialog.java}

\lstinputlisting{../DateDialog.java}

\subsection{DayDataModel.java}

\lstinputlisting{../DayDataModel.java}

\subsection{DayView.java}

\lstinputlisting{../DayView.java}

\subsection{Event.java}

\lstinputlisting{../Event.java}

\subsection{EventDialog.java}

\lstinputlisting{../EventDialog.java}

\subsection{EventRenderer.java}

\lstinputlisting{../EventRenderer.java}

\subsection{JEventField.java}

\lstinputlisting{../JEventField.java}

\subsection{Main.java}

\lstinputlisting{../Main.java}

\subsection{MainFrame.java}

\lstinputlisting{../MainFrame.java}

\subsection{MonthDataModel.java}

\lstinputlisting{../MonthDataModel.java}

\subsection{MonthView.java}

\lstinputlisting{../MonthView.java}

\subsection{Reminder.java}

\lstinputlisting{../Reminder.java}

\subsection{ReminderDialog.java}

\lstinputlisting{../ReminderDialog.java}

\subsection{Repetition.java}

\lstinputlisting{../Repetition.java}

\subsection{RepetitionDialog.java}

\lstinputlisting{../RepetitionDialog.java}

\subsection{SpacedPanel.java}

\lstinputlisting{../SpacedPanel.java}

\subsection{Time.java}

\lstinputlisting{../Time.java}

\subsection{WeekDataModel.java}

\lstinputlisting{../WeekDataModel.java}

\subsection{WeekView.java}

\lstinputlisting{../WeekView.java}

\subsection{YearDataModel.java}

\lstinputlisting{../YearDataModel.java}

\subsection{YearView.java}

\lstinputlisting{../YearView.java}

\end{document}

%%%%%%%%%%%%%%%%%%%%%%%%%%%%%%%%%%%%%%%%%%%%%%%%%%%%%%%%%%%%%%%%%%%%%%%
