\documentclass{article}

%%%%%%%%%%%%%%%%%%%%%%%%%%%%%%%%%%%%%%%%%%%%%%%%%%%%%%%%%%%%%%%%%%%%%%%

\usepackage{fullpage}
\usepackage{url}
\usepackage{graphicx}
\usepackage{color}
\definecolor{javared}{rgb}{0.6,0,0}
\definecolor{javagreen}{rgb}{0.25,0.5,0.35}
\definecolor{javapurple}{rgb}{0.5,0,0.35}
\definecolor{javadocblue}{rgb}{0.25,0.35,0.75}
\usepackage{listings}
\lstset {
	language=Java,
	basicstyle=\ttfamily,
	keywordstyle=\color{javapurple}\bfseries,
	stringstyle=\color{javared},
	commentstyle=\color{javagreen},
	morecomment=[s][\color{javadocblue}]{/**}{*/},
	numbers=left,
	numberstyle=\color{black},
	tabsize=4,
	showspaces=false,
	showstringspaces=false
}
%\usepackage{fullpage}

% Remove red border around refs and make them stand out.
\usepackage[colorlinks=true,linkcolor=blue]{hyperref}

%%%%%%%%%%%%%%%%%%%%%%%%%%%%%%%%%%%%%%%%%%%%%%%%%%%%%%%%%%%%%%%%%%%%%%%
%% Check these macro values for appropriateness for your own document.

\title{IS3 Group Report}

%%authors
\author{
  Richard Fleming \\
  James Gallagher \\
  Craig McLaughlin \\
  Victor Pantazi \\
  Gordon Reid \\
  Ross Taylor}

%%release date 
\date{\today}

%%%%%%%%%%%%%%%%%%%%%%%%%%%%%%%%%%%%%%%%%%%%%%%%%%%%%%%%%%%%%%%%%%%%%%%

\begin{document}

%%%%%%%%%%%%%%%%%%%%%%%%%%%%%%%%%%%%%%%%%%%%%%%%%%%%%%%%%%%%%%%%%%%%%%%

\maketitle

%%%%%%%%%%%%%%%%%%%%%%%%%%%%%%%%%%%%%%%%%%%%%%%%%%%%%%%%%%%%%%%%%%%%%%%
%% Standard section for all documents

\section{Overview}

Basic stuff about what we did...

\section{Specification}

Our GUI meets/doesn't meet the given specification...

\section{Evaluation of design}

To evaluate the design of the Swing prototype that the group produced, two main techneques were used.  

The first method the group decided to use for evaluating the design of the prototype was an heuristic evaluation. 
the group felt that a heuristic evaluation is a fast and cheap way to preform a usability inspection on the prototype, and would be a easy way to identify usability problems with the design of the system. 

one member of the team (James) was selected as an evaluator, and tasked with examing the interface of the prototype calendar and evaluating how it complies with recognized usability principles. 

The first thing the evaluator did was to select a collection of use cases, which he could then use to evaluate the extent to which the interface is likely to be compatible with the intended users’ needs. 

The use cases selected were : 
\begin{itemize}
\item Add an appointment.

\item 1.Change views from month, to week, to day 
\item 2.Add an appointment 
\item 3.Set a recurring appointment to occur once a week and finish in three months time
\item 4.Find the busiest and quietest days in the month
\item 5.Set different categories for university, social and job appointments
\end{itemize}

Afer this the evaluator decided on a set of heuristics on which the prototype should be judged. 

\subsection{Heuristic Evaluation}

The evaluator decided upon a set of heuristics on which the Swing prototype the team built should be judged.

The heuristics sellected were as follows : 

\item Visibility of system status : The system should always keep users informed about what is going on, through appropriate feedback within reasonable time. 

\item Match between system and the real world : The system should speak the users' language, with words, phrases and concepts familiar to the user, rather than system-oriented terms. Follow real-world conventions, making information appear in a natural and logical order.
 
\item User control and freedom : Users often choose system functions by mistake and will need a clearly marked "emergency exit" to leave the unwanted state without having to go through an extended dialogue. Support undo and redo. 

\item Consistency and standards : Users should not have to wonder whether different words, situations, or actions mean the same thing. Follow platform conventions. 

\item Error prevention : Even better than good error messages is a careful design which prevents a problem from occurring in the first place. Either eliminate error-prone conditions or check for them and present users with a confirmation option before they commit to the action. 

\item Recognition rather than recall : Minimize the user's memory load by making objects, actions, and options visible. The user should not have to remember information from one part of the dialogue to another. Instructions for use of the system should be visible or easily retrievable whenever appropriate. 

\item Flexibility and efficiency of use : Accelerators -- unseen by the novice user -- may often speed up the interaction for the expert user such that the system can cater to both inexperienced and experienced users. Allow users to tailor frequent actions. 
    
\item Aesthetic and minimalist design :Dialogues should not contain information which is irrelevant or rarely needed. Every extra unit of information in a dialogue competes with the relevant units of information and diminishes their relative visibility. 
    
\item Help users recognize, diagnose, and recover from errors : Error messages should be expressed in plain language (no codes), precisely indicate the problem, and constructively suggest a solution. 
    
\item Help and documentation : Even though it is better if the system can be used without documentation, it may be necessary to provide help and documentation. Any such information should be easy to search, focused on the user's task, list concrete steps to be carried out, and not be too large.    


Below are the findings from the evaluation:








\section{Comparison with paper prototype}

Say how much are GUI is the same as, or different to, the final paper
prototype. Describe any changes made and why. What would we like to have
added and how would it be done in swing.

\appendix

\section{Code listing}

\subsection{BaseView.java}

\lstinputlisting{../BaseView.java}

\subsection{CalendarModel.java}

\lstinputlisting{../CalendarModel.java}

\subsection{CalendarOperation.java}

\lstinputlisting{../CalendarOperation.java}

\subsection{CategoryDialog.java}

\lstinputlisting{../CategoryDialog.java}

\subsection{Date.java}

\lstinputlisting{../Date.java}

\subsection{DateDialog.java}

\lstinputlisting{../DateDialog.java}

\subsection{DayDataModel.java}

\lstinputlisting{../DayDataModel.java}

\subsection{DayView.java}

\lstinputlisting{../DayView.java}

\subsection{Event.java}

\lstinputlisting{../Event.java}

\subsection{EventDialog.java}

\lstinputlisting{../EventDialog.java}

\subsection{EventRenderer.java}

\lstinputlisting{../EventRenderer.java}

\subsection{JEventField.java}

\lstinputlisting{../JEventField.java}

\subsection{Main.java}

\lstinputlisting{../Main.java}

\subsection{MainFrame.java}

\lstinputlisting{../MainFrame.java}

\subsection{MonthDataModel.java}

\lstinputlisting{../MonthDataModel.java}

\subsection{MonthView.java}

\lstinputlisting{../MonthView.java}

\subsection{Reminder.java}

\lstinputlisting{../Reminder.java}

\subsection{ReminderDialog.java}

\lstinputlisting{../ReminderDialog.java}

\subsection{Repetition.java}

\lstinputlisting{../Repetition.java}

\subsection{RepetitionDialog.java}

\lstinputlisting{../RepetitionDialog.java}

\subsection{SpacedPanel.java}

\lstinputlisting{../SpacedPanel.java}

\subsection{Time.java}

\lstinputlisting{../Time.java}

\subsection{WeekDataModel.java}

\lstinputlisting{../WeekDataModel.java}

\subsection{WeekView.java}

\lstinputlisting{../WeekView.java}

\subsection{YearDataModel.java}

\lstinputlisting{../YearDataModel.java}

\subsection{YearView.java}

\lstinputlisting{../YearView.java}

\end{document}

%%%%%%%%%%%%%%%%%%%%%%%%%%%%%%%%%%%%%%%%%%%%%%%%%%%%%%%%%%%%%%%%%%%%%%%
